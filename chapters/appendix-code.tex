Here is our code. Bits over trees, courtesy of HEC!

\section{Pose Estimation}
% inspired by https://xkcd.com/221/
\begin{lstlisting}[language=python, showstringspaces=false,frame=single]
  # From Python
  # It requires OpenCV installed for Python
  import sys
  import cv2
  import os
  from sys import platform
  import argparse
  import numpy as np
  import csv

  #dir_path = os.path.dirname(os.path.realpath(__file__))
  dir_path = os.path.dirname('D:/Data/Kaavish I/OpenposeFiles/openpose/build/')

  video_name = 'video1_good'
  try:
      if platform == "win32":
          # Change these variables to point to the correct folder (Release/x64 etc.)
          sys.path.append(dir_path + '/../../python/openpose/Release');
          os.environ['PATH']  = os.environ['PATH'] + ';' + dir_path + '/../../x64/Release;' +  dir_path + '/../../bin;'
          import pyopenpose as op
  except ImportError as e:
      print('Error: OpenPose library could not be found. Did you enable `BUILD_PYTHON` in CMake and have this Python script in the right folder?')
      raise e



  # Custom Params (refer to include/openpose/flags.hpp for more parameters)
  params = dict()
  params["model_folder"] = "../../../models/"
  params["net_resolution"] = "-1x80"
  params["model_pose"] = "COCO"
  params["number_people_max"] = 1
  params["keypoint_scale"] = 3
  #params["disable_blending"] = True

  # Starting OpenPose
  opWrapper = op.WrapperPython()
  opWrapper.configure(params)
  opWrapper.start()

  data_path = 'D:/Data/Kaavish I/GitRepository/Virtual-Self-Defense-Trainer/Code/PoseEvaluation/videoData'
  file_data = []
  os.mkdir(data_path + '/../frames/' + video_name + '/')

  # Opens the Video file
  cap= cv2.VideoCapture(data_path + '/videos/' + video_name + '.mp4')
  i=0
  while(cap.isOpened()):
      ret, img = cap.read()
      if ret == False:
          break
      if (img.shape[1] != 640 or img.shape[0] != 480):
          width = 640
          height = 480
          dim = (width, height)
          # resize image
          img = cv2.resize(img, dim, interpolation = cv2.INTER_AREA)
      # Process Image
      datum = op.Datum()
      datum.cvInputData = img
      opWrapper.emplaceAndPop([datum])
      cv2.imshow("OpenPose 1.5.1 - Tutorial Python API", datum.cvOutputData)
      cv2.imwrite(data_path + '/../frames/' + video_name + '/' + str(i)+'.png',datum.cvOutputData)
      if datum.poseKeypoints.shape == (1,18,3):
          file_data.extend(datum.poseKeypoints[0])
      else:
          file_data.extend(np.zeros(shape=(18,3)))
      i+=1
      key = cv2.waitKey(1)
      if key==ord('q'):
          break

  with open(data_path + '/csv/' + video_name + '.csv', 'w',newline='') as file:
      writer = csv.writer(file, delimiter=',')
      writer.writerows(file_data) 
    
  cap.release()
  cv2.destroyAllWindows()

\end{lstlisting}

% Alternately...
Our code can be found at \href{https://github.com/habib-university/Kaavish-Template}{this GitHub link}.

%%% Local Variables:
%%% mode: latex
%%% TeX-master: "../report"
%%% End:
