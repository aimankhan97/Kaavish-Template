\section{Manual of Prototype}

In order to setup the software the person must setup openpose first and get it to running. The link \footnote{https://github.com/CMU-Perceptual-Computing-Lab/openpose/blob/master/doc/installation.md} helps with the installation. Make sure each and every step is followed, setting up open pose is highly critical and even if a single step is missed it will not work. After open pose has been setup an RGB camera is to be connected to computer and .py files in appendix need to be run making some changes to make sure the correct webcam is being detected and used. For user interface unity needs to be setup loading all the unity scripts project included here. This is all one needs to do for the software side. If everything is done correctly the unity project once run will execute everything on its own. 

For hardware side the electrodes need to be connected to the T-blocks and Arduino UNO to the SIL connectors. The electrode connections must be made properly noting down, from the circuit diagram in figure \ref{fig:NodeCircuit}, which digital I/O pin does each T-block correspond to. Changes be made in the arduino code according to the connection made. This is a flexible step, the digital pins can be assigned any way making sure these correspond to the connection made. Following this the wearable we worn and arduino cable be connected to the computer. When unity is running it interfaces with the arduino automatically. 

The scheme of the placement of the electrodes on the muscles should be followed strictly as shown in the Figures \ref{fig:spatialCal} and \ref{fig:wearable} for desirable effects, otherwise it may lead to uncomfortable results.


\subsection{Disclaimer}
EMS, though an extremely resourceful entity in haptics, has several safety precautions that need to be taken into account. EMS uses slight current on the user’s body to simulate “skin receptors and muscle fibers”, hence there comes forth several hardware and software precautions users need to take to ensure safety \cite{toolkit}. The idea of setting up a calibration system for the user before proceeding to using EMS for the rest of the program is derived from the varying levels of current from person to person. The calibration system guarantees safety of the user and hardware; asking the user for their preferred level removes the chance of any disruption or damage. Keeping precautions in our mind, EMS restricts its users to those who it will not harm in any way. People who have a heart disease, pregnant women, people with epilepsy, skin disease or cancer, and those who have recently gone through surgery should not use EMS devices.

Also, take care in not to place the electrodes near the heart, neck and face. Keep an active device away from children.
