\chapter*{Executive Summary}
Crime-rate and acts of physical violence in third world countries like Pakistan are alarmingly high especially acts of violence against women. Therefore, the need to provide self defense training arises in such societies. Self-defense training centers are not accessible to most of the people in Pakistan, either because people are not allowed to attend these institutes or there is no such institutes in their locality. Online courses for self-defense training are non-interactive and do not offer any feedback to the performed moves.


In this project a virtual trainer is developed. The user interacts with a virtual trainer via haptic feedback and user interface. Using low power electric pulses of different shapes and wave characteristics, muscles are stimulated. The user movements are captured using a camera. A stick figure of the user is generated using the OpenPose library, an open source platform that uses convolutional neural network. The keypoints of joints are extracted, the poses are compared with bench mark poses and scores of the performance are computed. The scores are then mapped onto natural language feedback. The poses will also be used to trigger the haptic device appropriately. A major challenge is the varying anatomies of people. A single electrical signal characteristic cannot work for all the people. Therefore, a calibration protocol is needed, via which each user can set the desired signal parameters from a set of presets. The calibration is being done in signal characteristics as well as in the spatial placement of electrodes.

\begin{flushleft}
\textbf{KEYWORDS:} Haptics, self-defense, pose evaluation, virtual training, scoring
\end{flushleft}